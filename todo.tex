\documentclass[10pt,a4paper]{article}
\usepackage[utf8]{inputenc}
\usepackage[english]{babel}
\usepackage{amsmath}
\usepackage{amsfonts}
\usepackage{amssymb}
\usepackage{graphicx}

\author{Linnéa Claesson}

\begin{document}
Läs igenom Johans kommentarer som han mailade ut.

\section*{System Requirements}
Shall och inte should i kraven. 

\begin{description}
\item[Introduction] \textit{Linnéa}

\item[Background] \textit{Linnéa}

\item[Data model] Mer utförlig text och förklaringar - \textit{Billy}

\item[Data dictionary] Förklaring av orden vi använder, framförallt i data model - \textit{Billy}

\item[Terminology/ordlista] Som inte är samma som data dictionary. - \textit{Emma}

\item[Context diagram] Flytta till Domain och lägg till mer förklaringar, både i bilden och text till bilden. Förklara roller och interface. - \textit{Caroline}

\item[Stakeholders] Mer text, vilka är dem. Hur viktiga är dem etc? Koppla samman med context diagram Lägga till även developers? Prioritering. - \textit{Caroline}

\item[Task descriptions] Framförallt på domännivå. Hur går det till när man skickar kort? Betala? Designar vykort? Etc. - \textit{Jacob}

\item[Designkrav] Lägg in bilder från prototyper, beskriva prototyperna. - \textit{Johan}

\item[Datakrav] Virtual window vid skapandet av vykort, vilken data ska skickas in då? - \textit{Billy} 

\item[Prioritering av krav] Börja titta på prioritering och även gruppering av krav, vilka hänger ihop? Gör ett quality grid (tex).  - \textit{Carl} 

\item[QUPER] Gör en QUPER analys av tex bildgalleriet (jämför med svenska appen) - \textit{Carl}

\item[Featurekrav] Se Johans kommentarer och länkar, behöver brytas ner - \textit{Emma \& Linnéa} 

\item[Validation checklist] \textit{Johan}

\item[Release Plan] Inför release 3

\end{description}

\section*{Project Experiences}
Här är det viktigt att alla skriver in sina erfarenheter och reflektioner kontinuerligt under projektet. Tänk på att vara tydliga, förklara allting noga och motivera väl varför vi gjort/inte gjort som vi gjorde. Använd passiv form (inte vi).

Titta igenom det ni hittills skrivit här också och komplettera, se Johans kommentarer och tänk på att skriva noggranna motiveringar. 





\end{document}
\documentclass[10pt,a4paper]{article}
\usepackage[utf8]{inputenc}
\usepackage[english]{babel}
\usepackage{amsmath}
\usepackage{amsfonts}
\usepackage{amssymb}
\usepackage{graphicx}

\title{Questions, w4}
\author{\textbf{Team C}\\
Emma Albertz\\
Caroline Brandberg\\
Linnéa Claesson\\
Billy Joansson\\
Johan Ju\\
Jacob Mejvik\\
Carl Rynegardh}

\begin{document}
\maketitle
\section*{Chapter 1}
\subsection*{Problem 1: Terminology}
\begin{description}
\item[Proposition] It is recommended that validation is first done in the acceptance test.
\item[Reason] Validation means checking that the product fulfils the requirements.
\item[Correct answer] E (Both the proposition and the reason are false.)
\item[Motivation] The proposition is false as it is the verification that is in minimum done in
an acceptance test. The reason is false as it is the description for verification that is
described. Validation means the customer’s check that requirements match demands. Ver-
ification means checking that the product fulfills the requirements. Thereby the change of
the word ”Validation” to ”Verification” in both the proposition and for the reason would
have resulted in correct statements.
\item[Reference] Lau: p. 3, 6
\item[Learning objective] 1
\item[Main responsible] Caroline Brandberg
\end{description}

\subsection*{Problem 2: Business goals}
\begin{description}
\item[Proposition] Business goals help the customer as well as the supplier to convince themselves that the requirements help the customer to reach their goals. 
\item[Reason] Business goals describe why the customer is willing to pay for the system. 
\item[Correct answer] A (Both the proposition and the reason are correct statements, AND the reason explains the proposition in a correct way.)
\item[Motivation] It is important to know why the system is developed.
\item[Reference] Lau: p. 24
\item[Learning objective] 6
\item[Main responsible] Emma Albertz
\end{description}

\section*{Chapter 2}
\subsection*{Problem 1: Data model}
\begin{description}
\item[Proposition] Using a data model is a good technique for customers without technical knowledge from a validation perspective. 
\item[Reason] E/R-model is a type of data model.
\item[Correct answer] D (The proposition is false, but the reason is a true statement.)
\item[Motivation] To understand E/R-modelling you need technical knowledge in the field, which makes it difficult to use for customers without technical knowledge. E/R-model is one of several types of data models. 
\item[Reference] Lau: p. 44, 55
\item[Learning objective] 3
\item[Main responsible] Billy Johansson
\end{description}

\section*{Chapter 3 \& 4}
\subsection*{Problem 1: Tasks}
\begin{description} 
\item[Proposition] A good user task should be "closed", which means it should not be possible to add additional functionality. 
\item[Reason] It is important that the user considers he has achieved something with the task. 
\item[Correct answer] D (The proposition is false, but the reason is a true statement.)
\item[Motivation] A good task should be closed, \textit{but} this means that the task should have some sort of "closure", i.e. finish with a meaningful goal achieved. 
\item[Reference] Lau: p. 116
\item[Learning objective] 3
\item[Main responsible] Jacob Mejvik
\end{description}

\subsection*{Problem 2: Decision table}
\begin{description}
\item[Proposition] Decision tables are good for describing what decisions the program shall make.
\item[Reason] Decision tables are shown as a state machine that for each state and input gives the viewer an output that says what the user shall do.
\item[Correct answer] C (The proposition is true, but the reason is false.)
\item[Motivation] The decision tables do help the developer with what the business rules are but it shows the rules as a table and not a state machine.
\item[Reference] Lau: Chapter 4 pages 161-163
\item[Learning objective] 1, 3
\item[Main responsible] Johan Ju
\end{description}

\section*{Chapter 8}
\subsection*{Problem 1: Task demonstration}
\begin{description}
\item[Proposition] Task demonstration is good for finding present problems.
\item[Reason] In many cases, users cannot explain what they do in their daily work.
\item[Correct answer] A (Both the proposition and the reason are correct statements, AND the reason explains the proposition in a correct way.)
\item[Motivation] If users cannot explain what they do in their daily work it is hard to realize what problems of the current system could be. By watching users do their work problems could be identified.
\item[Reference] Lau: p. 338, 341
\item[Learning objective] 2, 3
\item[Main responsible] Carl Rynegardh
\end{description}

\subsection*{Problem 2: Brainstorming}
\begin{description}
\item[Proposition] Brainstorming is a preferred elicitation technique for eliciting realistic possibilities.
\item[Reason] People can come up with a lot of ideas and suggestions without being judged or ridiculed.
\item[Correct answer] D (The proposition is false, but the reason is a true statement.)
\item[Motivation] Brainstorming is good for coming up with a lot of new ideas, since it creates an environment where people can (and should) say all ideas that come to mind to inspire creativity. It is therefore not a preferred technique for coming up with realistic possibilities, since it often generates a lot of not so good ideas.
\item[Reference] Lau: p. 338, 342-343
\item[Learning objective] 1, 3, 4
\item[Main responsible] Linnéa Claesson
\end{description}


\end{document}
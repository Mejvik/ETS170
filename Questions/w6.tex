\documentclass[10pt,a4paper]{article}
\usepackage[utf8]{inputenc}
\usepackage[english]{babel}
\usepackage{amsmath}
\usepackage{amsfonts}
\usepackage{amssymb}
\usepackage{graphicx}

\title{Questions, w6}
\author{\textbf{Team C}\\
Emma Albertz\\
Caroline Brandberg\\
Linnéa Claesson\\
Billy Joansson\\
Johan Ju\\
Jacob Mejvik\\
Carl Rynegardh}

\begin{document}
\maketitle
\section*{Chapter 5, 7}
\subsection*{Problem 1: Interface}
\begin{description}
\item[Proposition] When specifying the detailed interface, several things are needed. One of those things are the messages. 
\item[Reason] The messages is for example the data involved, how the message is identified or event communicated.  
\item[Correct answer] A (\textit{Both the proposition and the reason are correct statements,
AND the reason explains the proposition in a correct way.})
\item[Motivation]  The proposition is true as there are four things mentioned in Lau needed to be specified the detailed interface. These are the physical channel, the message, the protocol and the semantics. The reason is also true since those questions are presented to let the user know what Lau means by "The message" in the book.
\item[Reference] Lau: Chapter 5 page 214
\item[Learning objective] 4 
\item[Main responsible] Caroline Brandberg
\end{description} 

\begin{description}
\item[Proposition] Project inception is included in the product life cycle of the requirements. 
\item[Reason] This is needed let the customer specify how he measures the requirements. 
\item[Correct answer] C (\textit{The proposition is true, but the reason is false.})
\item[Motivation] The proposition is true, since this is where question such as what should be included/excluded, advantages of the system, who is affected and many more vague issues are stated. The reason is false since that is relating to the part "Rating the requirements" rather than "inception". 
\item[Reference] Lau: Chapter 7 pages 292, 305
\item[Learning objective] 5
\item[Main responsible] Caroline Brandberg
\end{description}

\section*{Chapter 6, QUPER}
\subsection*{Problem 2: TITLE OF PROBLEM}
\begin{description}
\item[Proposition] 
\item[Reason] 
\item[Correct answer]
\item[Motivation]
\item[Reference]
\item[Learning objective]
\item[Main responsible] Billy Johansson
\end{description}

\begin{description}
\item[Proposition] 
\item[Reason] 
\item[Correct answer]
\item[Motivation]
\item[Reference]
\item[Learning objective]
\item[Main responsible] Billy Johansson
\end{description}

\subsection*{Problem 3: TITLE OF PROBLEM}
\begin{description}
\item[Proposition] 
\item[Reason] 
\item[Correct answer]
\item[Motivation]
\item[Reference]
\item[Learning objective]
\item[Main responsible] Emma Albertz
\end{description}

\begin{description}
\item[Proposition] 
\item[Reason] 
\item[Correct answer]
\item[Motivation]
\item[Reference]
\item[Learning objective]
\item[Main responsible] Emma Albertz
\end{description}

\section*{Chapter 9, INSP}
\subsection*{Problem 4: TITLE OF PROBLEM}
\begin{description}
\item[Proposition] In practice, Unambiguous specs are one of the biggest problems. 
\item[Reason] If a developer does not understand an requirement, he/she normally just guess what the customer wants.
\item[Correct answer] E
\item[Motivation] Unambiguous specs are not a big problem in practice because developers normally ask customers to clarify. It can be a problem if the developer believes he knows, even though the customer wants something else. However, it is normally not a big problem.
\item[Reference] p. 376
\item[Learning objective] 1
\item[Main responsible] Carl Rynegardh
\end{description}

\subsection*{Problem 4: TITLE OF PROBLEM}
\begin{description}
\item[Proposition] Using checklists for content checks can be a good idea.
\item[Reason] A checklist is a way of reminding you what should be in the spec.
\item[Correct answer] A
\item[Motivation] A content check, looks at the content to make sure everything is in there. Checklists can be used to check that the spec contains what should be in it. You can check of one item at a time.
\item[Reference] p. 382-384
\item[Learning objective] 1,2
\item[Main responsible] Carl Rynegardh
\end{description}

\subsection*{Problem : TITLE OF PROBLEM}
\begin{description}
\item[Proposition] 
\item[Reason] 
\item[Correct answer]
\item[Motivation]
\item[Reference]
\item[Learning objective]
\item[Main responsible] Jacob Mejvik
\end{description}

\section*{MDRE, PRIO, RP}
\subsection*{Problem : TITLE OF PROBLEM}
\begin{description}
\item[Proposition] 
\item[Reason] 
\item[Correct answer]
\item[Motivation]
\item[Reference]
\item[Learning objective]
\item[Main responsible] Jacob Mejvik
\end{description}

\subsection*{Problem : TITLE OF PROBLEM}
\begin{description}
\item[Proposition] 
\item[Reason] 
\item[Correct answer]
\item[Motivation]
\item[Reference]
\item[Learning objective]
\item[Main responsible] Johan Ju
\end{description}

\subsection*{Problem : TITLE OF PROBLEM}
\begin{description}
\item[Proposition] 
\item[Reason] 
\item[Correct answer]
\item[Motivation]
\item[Reference]
\item[Learning objective]
\item[Main responsible] Johan Ju
\end{description}


\section*{AGRE, INTDEP}
\subsection*{Problem : TITLE OF PROBLEM}
\begin{description}
\item[Proposition] 
\item[Reason] 
\item[Correct answer]
\item[Motivation]
\item[Reference]
\item[Learning objective]
\item[Main responsible] Linn\'ea Claesson
\end{description}

\subsection*{Problem : TITLE OF PROBLEM}
\begin{description}
\item[Proposition] 
\item[Reason] 
\item[Correct answer]
\item[Motivation]
\item[Reference]
\item[Learning objective]
\item[Main responsible] Linn\'ea Claesson
\end{description}




\end{document}
\documentclass[10pt,a4paper]{article}
\usepackage[utf8]{inputenc}
\usepackage[english]{babel}
\usepackage{amsmath}
\usepackage{amsfonts}
\usepackage{amssymb}
\usepackage{graphicx}

\author{Linnéa Claesson}

\begin{document}

\section*{19 november}
{\Large Följande ska vara klart \textbf{söndag kl 9.00}:}\\

\textbf{Ska med i System Requirements:}
\begin{itemize}
\item Context diagram + text. Lägg till bildleverantör, betalningssytem, stakeholders, nyckelkunden är mellanled till supplier. De har en skrivare och postar vykorten (fråga Jacob om oklart). - \textit{Caroline}
\item Resultat/diagram + text från enkäterna - \textit{Caroline}
\item Konkreta krav från kundmötet - \textit{Jacob och Emma}
\item Main goals - \textit{alla tar fram, Linn\'ea skriver in}
\item E/R-diagram + text - \textit{Billy}
\item Functional requirements från PMv2/Enkäterna - \textit{Carl}

\end{itemize}

\textbf{Ska med i Project Experiences:}
\begin{itemize}
\item Reflektioner kring context diagram - \textit{Caroline}
\item Reflektioner kring resultat/diagram från enkäterna - \textit{Caroline}
\item Hur var kundkontakten? - \textit{E\&J}
\item Reflektioner kring E/R-diagram - \textit{Billy}
\item Prototyper, reflektioner - \textit{Johan}
\item Enkät - \textit{Carl}
\item Dokumentanalys - \textit{Carl}

\end{itemize}

\section*{16 november}

\begin{itemize}
\item Linnéa delar upp vilka som ska göra frågor till vilka kapitel.
\item Inför release 1:
\begin{description}
\item[Project Experiences] Linnéa fixar dokument-mall, alla ansvarar för att skriva in reflektioner kring sina respektive delar.
\item[System Requirements] Linnéa fixar dokument-mall, alla ansvarar för att sina krav kommer in (från reqT).
\item[End User] Caroline, Johan, Carl och Billy gör prototyper, dokumentanalys och enkät/intervjuer.
\item[Supplier] Emma och Jakob kontaktar existerande företag och gör en fokusgrupp med nyckelkunden.
\end{description}
\item Ses på torsdag kl 8 för att se över dokumentet. Inför detta möte ska alla ha skrivit in relevant text och krav.
\end{itemize}

\section*{9 november}
\begin{itemize}
\item Managern har ansvar över sitt område, men alla ska bidra till allt.
\item Versionsnumrering: v0.01 -> v0.02 -> ... -> v0.09 -> v0.10 osv. Glöm inte att uppdatera version (och datum) varje gång du gör ändringar i ett dokument.
\item Commit message i git: Nytt versionsnummer och vad du gjort.
\end{itemize}
\end{document}
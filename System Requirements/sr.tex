% INNAN DU COMMITAR!
% Uppdatera datum
% Uppdatera version
%-----
% Document name



\documentclass[10pt,a4paper]{article}
\usepackage[utf8]{inputenc}
\usepackage[english]{babel}
\usepackage{amsmath}
\usepackage{amsfonts}
\usepackage{amssymb}
\usepackage{graphicx}
\usepackage{geometry}

\title{PostCardBuddy}
\author{Team C}

\begin{document}
\begin{titlepage}
\newgeometry{left=2cm,top=1cm,right=2cm}
\newcommand{\HRule}{\rule{\linewidth}{0.5mm}}


\begin{flushright}
November 16, 2015 v0.02\\[3cm]
\end{flushright}


\centering
\textsc{\LARGE Team C}\\[0.5cm]

\HRule \\[0.4cm]
{ \huge \bfseries PostCardBuddy}\\[0.3cm]
{\Large \bfseries System Requirements}\\[0.4cm] % Title of your document
\HRule \\[1.5cm]

\vfill
\begin{flushleft}
%Authors, write on separate lines
\textit{Authors of this document:}\\
Emma Albertz\\
Caroline Brandberg\\
Linnéa Claesson\\
Billy Johansson\\
Johan Ju\\
Jacob Mejvik\\
Carl Rynegardh
\end{flushleft}

\end{titlepage}
\pagenumbering{gobble}



%\begin{center}
%\textit{\large Version History}
%
%    \begin{tabular}{ | l | l | l | p{5cm} |}
%    \hline
%    \textbf{Version} & \textbf{Date} & \textbf{Responsible} & \textbf{Description} \\ \hline
%    1.0 & 2015-10-14 & EA, LC & Baseline\\ \hline
%    \end{tabular}
%\end{center}



\setcounter{tocdepth}{2}
\tableofcontents
\newpage
\pagenumbering{arabic}


%--------------------------------------------------------------------%
%----------------- System Requirements ------------------------------%
%--------------------------------------------------------------------%
% Different types of system requirements (e.g. data, function, quality) at different levels (e.g. goal, domain, product, design).
% Each requirement should have a unique identity (name or number) that is consistent between releases.
% A subset of the requirements should be prioritized.
\section{System Requirements}

\subsection{Data Requirements}
\subsubsection{Goal}
\subsubsection{Domain}
\subsubsection{Product}
\subsubsection{Design}

\subsection{Function Requirements}
\subsubsection{Goal}
\subsubsection{Domain}
\subsubsection{Product}
\subsubsection{Design}

\subsection{Quality Requirements}
\subsubsection{Goal}
\subsubsection{Domain}
\subsubsection{Product}
\subsubsection{Design}


%--------------------------------------------------------------------%
%--------------- Specification Techniques ---------------------------%
%--------------------------------------------------------------------%
% Several different specification techniques (e.g. context diagrams, features, virtual windows, task descriptions).
\section{Specification Techniques}
\subsection{Context Diagrams}
\subsection{Features}
\subsection{Virtual Windows}
\subsection{Task Descriptions}


%--------------------------------------------------------------------%
%------------------ Release Plan ------------------------------------%
%--------------------------------------------------------------------%
% A release plan defining which requirements that are implemented in each of three releases, namely R3 (final release for this course project), and the imagined future releases R4 and R5. The release plan shall include information used to derive the plan such as priorities and cost.
% The requirements planned for R3 (in the release plan) should be implemented as mock-up designs (e.g. screens and prototypes, analog drawings, clickable presentations, executable gui mockups) and thus included in the final (R3) release.
\section{Release Plan}




\end{document}


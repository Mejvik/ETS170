\documentclass[10pt,a4paper]{article}
\usepackage[utf8]{inputenc}
\usepackage[english]{babel}
\usepackage{amsmath}
\usepackage{amsfonts}
\usepackage{amssymb}
\usepackage{graphicx}

\author{Linnéa Claesson}
\title{Kundmöte 1}

\begin{document}
\maketitle

Följande punkter kom upp under första mötet med kunden:

\begin{itemize}
\item Kunden är \textbf{entreprenörer}, vi tillhandahåller skrivare och bestämmer format. De sköter underhåll av skrivaren.

\item Information om mottagare till kunden. Gränssnitt för information i en .xml-fil. 

\item Kunden har en central med en eller flera skrivare och skickar alla vykort till leverantören därifrån.

\item Frankering, kunden fixar själva avtal med leverantör av vykort angående \textbf{förtryckt porto}.

\item Leveranstid 24 h + leverantörens leveranstid.

\item \textbf{Notifikation}, nu har leverantören fått ditt vykort. Kräver avstämningssytem för när vykorten gått iväg.

\item \textbf{Målgrupp}: turister, unga, barnbarn.

\item Stöd för Android och iOS. Web-baserat eventuellt i en senare release.

\item Om användaren saknar internet tillfälligt ska vykorten sparas i en \textbf{cache} för att sen skickas när användaren får tillgång till internet igen.

\item Användarvänlig och \textbf{intuitiv} applikation.

\item \textbf{Betalning} - kreditkort (registrera kort i appen)/faktura/paypal/mobilfaktura/mottagaren betalar.
\end{itemize}
\end{document}
\documentclass[10pt,a4paper]{article}
\usepackage[utf8]{inputenc}
\usepackage[english]{babel}
\usepackage{amsmath}
\usepackage{amsfonts}
\usepackage{amssymb}
\usepackage{graphicx}
\author{Linnéa Claesson}
\begin{document}
{\huge Allmänt att tänka på:}
\begin{itemize}
\item A description of our requirements engineering work, including experiences and reflections in relation to learning objectives.
\item The Project Experiences should not include course evaluation issues, but focus on your own work and learning outcome.
\item Under \textbf{Methods and Techniques}, beskriv metoden/tekniken och varför den valdes.
\item Under \textbf{Reflections}, utvärdera hur metoden/tekniken som användes fungerade (se kommentarer i pe.tex eller kursprogrammet). Vad var bra/dåligt med den? Vad har vi lärt oss? Skulle vi använda den i verkligheten, etc?
\item \textbf{Andvänd då-tid}, rapporten lämnas in som helhet när projektet är klart och inte mellan varje release, dvs när den är klar kommer r1, r2 och r3 befinna sig i då-tiden och r4 och r5 i framtiden.
\end{itemize}


\section{Introduction}

\section{Methods and Techniques}
\subsection{Elicitation}
\begin{description}
\item[Brainstorming] 
\item[Questionnaire]
\item[Interviews]
\item[Prototypes]
\item[Document study]
\item[Data model] Hur och varför?
\item[Data dictionary/Virtual windows] Hur/varför användes den för elicitation?
\end{description}

\subsection{Specification}
{\Large Saknas:}
\begin{itemize}
\item[4B)] use at least four different specification techniques adequately tailored to the context.
\end{itemize}

\begin{description}
\item[Context diagram] 
\item[Data model] Hur/varför användes den för specification?
\item[Data dictionary/Virtual windows] Hur/varför användes den för specification?
\end{description}

\subsection{Validation}
{\Large Saknas:}
\begin{itemize}
\item[3G)] apply more than one validation technique.
\item[4G)] adapt the validation to the context and provide rationale for the chosen validation techniques.
\end{itemize}

\begin{description}
\item[Prototypes]
\end{description}

\subsection{Prioritization}
{\Large Saknas:}
\begin{itemize}
\item[3I)] use more than one prioritization technique in a relevant way.
\end{itemize}

\begin{description}
\item[Stakeholder]
\end{description}

\section{Reflections}
Reflection on the usage of these methods/techniques in terms of what was successful and what was challenging. Example questions for reflection: \textbf{What have you learned in relation to the learning objectives in this course program? What would you have done differently if you would do this project again as a "real" project, based on what you know now?}

{\Large Saknas:}
\begin{itemize}
\item[5B)] provide motivated estimations of target quality levels using well-defined scales.
\item[4F)] to find, prioritize and discuss requirements quality problems of different types, while reaching 	beyond form issues.
\item[5D)] reason about the relation between requirements quality problems and risks, both from a customer and developer viewpoint.
\end{itemize}
\subsection{Elicitation}
{\Large Saknas:}
\begin{itemize}
\item[ 4E)] reason about the need for further elicitation in relation to specification quality.
\end{itemize}
\begin{description}
\item[Brainstorming] 
\item[Questionnaire]
\item[Interviews] 
\item[Prototypes] 
\item[Document study]
\item[Data model]
\item[Data dictionary/Virtual windows]
\end{description}

\subsection{Specification}
{\Large Saknas:}
\begin{itemize}
\item[5A)] combine specification techniques in an explicitly motivated trade off between qualities and costs, where a high degree of specification completeness is achieved for a carefully selected subset of requirements.
\end{itemize}

\begin{description}
\item[Context diagram]
\item[Data model]
\item[Data dictionary/Virtual windows]
\end{description}

\subsection{Validation}
{\Large Saknas:}
\begin{itemize}
\item[3H)] reflect on validation experiences.
\end{itemize}
\subsection{Prioritization}
\section{Personal Statements}
{\Large Saknas från samtliga!}
\end{document}
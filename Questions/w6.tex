\documentclass[10pt,a4paper]{article}
\usepackage[utf8]{inputenc}
\usepackage[english]{babel}
\usepackage{amsmath}
\usepackage{amsfonts}
\usepackage{amssymb}
\usepackage{graphicx}

\title{Questions, w6}
\author{\textbf{Team C}\\
Emma Albertz\\
Caroline Brandberg\\
Linnéa Claesson\\
Billy Joansson\\
Johan Ju\\
Jacob Mejvik\\
Carl Rynegardh}

\begin{document}
\maketitle
\section*{Chapter 5, 7}
\subsection*{Problem 1: Interface}
\begin{description}
\item[Proposition] When specifying the detailed interface, several things are needed. One of those things are the messages. 
\item[Reason] The messages are for example the data involved, how the message is identified or event communicated.  
\item[Correct answer] A (\textit{Both the proposition and the reason are correct statements,
AND the reason explains the proposition in a correct way.})
\item[Motivation]  The proposition is true as there are four things mentioned in Lau needed to be specified the detailed interface. These are the physical channel, the message, the protocol and the semantics. The reason is also true since those questions are presented to let the user know what Lau means by "The message" in the book.
\item[Reference] Lau: Chapter 5 page 214
\item[Learning objective] 4 
\item[Main responsible] Caroline Brandberg
\end{description} 

%\begin{description}
%\item[Proposition] Project inception is included in the product life cycle of the requirements. 
%\item[Reason] This is needed to let the customer specify how he measures the requirements. 
%\item[Correct answer] C (\textit{The proposition is true, but the reason is false.})
%\item[Motivation] The proposition is true, since this is where question such as what should be included/excluded, advantages of the system, who is affected and many more vague issues are stated. The reason is false since that is relating to the part "Rating the requirements" rather than "inception". 
%\item[Reference] Lau: Chapter 7 pages 292, 305
%\item[Learning objective] 5
%\item[Main responsible] Caroline Brandberg
%\end{description}

\section*{Chapter 6, QUPER}
\subsection*{Problem 2: Quality Grid}
\begin{description}
\item[Proposition] The quality grid can be used to weed out unimportant quality factors of the project before writing requirements.
\item[Reason] It's a grid with intersections of criticality and quality factors, e.g Reliability : Important, Correctness : Ignore.
\item[Correct answer] A (\textit{Both the proposition and the reason are correct statements,
AND the reason explains the proposition in a correct way.})
\item[Motivation] The reason is true because that is a general description of a quality grid. The result of the quality grid is then the answer to the question "How important is this quality factor for my project?" which means that the proposition is true. 
\item[Reference] Lau: Chapter 6 pages 226-227.
\item[Learning objective] 3
\item[Main responsible] Billy Johansson
\end{description}

%\begin{description}
%\item[Proposition] Open target or open metric should never be used for design-level requirements. 
%\item[Reason] The customer should not be concerned about details on the design level. 
%\item[Correct answer] E (\textit{Both the proposition and the reason are false})
%\item[Motivation] The proposition is false because even if the results of an open target or open metric is not reasonable, they can still be discussed after. The reason is false because the customer might have valuable knowledge about the domain.
%\item[Reference] Lau: Chapter 6 pages 228-231.
%\item[Learning objective] 5
%\item[Main responsible] Billy Johansson
%\end{description}

\subsection*{Problem 3: TITLE OF PROBLEM}
\begin{description}
\item[Proposition] 
\item[Reason] 
\item[Correct answer]
\item[Motivation]
\item[Reference]
\item[Learning objective]
\item[Main responsible] Emma Albertz
\end{description}

\subsection*{Problem 3: TITLE OF PROBLEM}
\begin{description}
\item[Proposition] 
\item[Reason] 
\item[Correct answer]
\item[Motivation]
\item[Reference]
\item[Learning objective]
\item[Main responsible] Emma Albertz
\end{description}

\section*{Chapter 9, INSP}
%\subsection*{Problem 4: TITLE OF PROBLEM}
%\begin{description}
%\item[Proposition] In practice, unambiguous specifications are one of the biggest problems. 
%\item[Reason] If a developer does not understand a requirement, he/she normally just guess what the customer wants.
%\item[Correct answer] E (Both the proposition and the reason are false)
%\item[Motivation] Unambiguous specs are not a big problem in practice because developers normally ask customers to clarify. It can be a problem if the developer believes he knows, even though the customer wants something else. However, it is normally not a big problem.
%\item[Reference] p. 376
%\item[Learning objective] 1
%\item[Main responsible] Carl Rynegardh
%\end{description}

\subsection*{Problem 4: Check lists}
\begin{description}
\item[Proposition] Using check lists for content checks can be a good idea.
\item[Reason] A check list is a way of reminding you what should be in the spec.
\item[Correct answer] A (Both the proposition and the reason are correct statements,
AND the reason explains the proposition in a correct way.)
\item[Motivation] A content check, looks at the content to make sure everything is in there. Check lists can be used to check that the spec contains what should be in it. You can check of one item at a time.
\item[Reference] p. 382-384
\item[Learning objective] 1,2
\item[Main responsible] Carl Rynegardh
\end{description}

\subsection*{Problem : TITLE OF PROBLEM}
\begin{description}
\item[Proposition] 
\item[Reason] 
\item[Correct answer]
\item[Motivation]
\item[Reference]
\item[Learning objective]
\item[Main responsible] Jacob Mejvik
\end{description}

\section*{MDRE, PRIO, RP}
\subsection*{Problem : TITLE OF PROBLEM}
\begin{description}
\item[Proposition] 
\item[Reason] 
\item[Correct answer]
\item[Motivation]
\item[Reference]
\item[Learning objective]
\item[Main responsible] Jacob Mejvik
\end{description}

\subsection*{Problem 6: MDRE}
\begin{description}
\item[Proposition] Fulfilling contracts is the most important aspect of MDRE.
\item[Reason] Fulfilling contracts may increase costumer satisfaction. 
\item[Correct answer] D (The proposition is false, but the reason is a true statement.)
\item[Motivation] In MDRE the primary task is to deliver a product that costumer wants in time and often don't have a contract with costumer beforehand.
\item[Reference] ”Market-Driven Requirements Engineering for Software Products”, Björn
Regnell and Sjaak Brinkkemper p. 290-291
\item[Learning objective] 5, 6
\item[Main responsible] Johan Ju
\end{description}

%\subsection*{Problem : \$100}
%\begin{description} 
%\item[Proposition] It is recommended to use \$100 method iteratively.
%\item[Reason] The stakeholders will distribute the money more as their real prioritization after each iteration.  
%\item[Correct answer] E (Both the proposition and the reason are false)
%\item[Motivation] Some stakeholders may try to focus on something that is only useful for themselves when they know that other requirements will get money regardless. 
%\item[Reference] ”Requirements Prioritization”, Patrik Berander and Anneliese Andrews Chapter 4.4.2  
%\item[Learning objective] 1, 3
%\item[Main responsible] Johan Ju
%\end{description}


\section*{AGRE, INTDEP}
\subsection*{Problem 7: Interdependencies}
\begin{description}
\item[Proposition] The time and effort required to do pairwise assessment of requirements (i.e. finding possible interdependencies between them) is constant no matter how many requirements one have to assess. 
\item[Reason] Identifying singular requirements can reduce the time and effort needed to do pairwise assessment of a set of requirements. 
\item[Correct answer] D (The proposition is false, but the reason is a true statement)
\item[Motivation] The time and effort needed heavily increases with the number of requirements. The number of assessments are 
\begin{equation}
\sum_{1}^{n-1}i = \frac{n(n-1)}{2} 
\end{equation}
where n is number of requirements. The proposition is therefore false. The reason is a true statement though, since finding the singular requirements usually is a fast and easy task and then the number of assessments are reduced to
\begin{equation}
\sum_{1}^{n-1-s}i = \frac{(n-s)(n-1-s)}{2} 
\end{equation}
The reason is therefore a true statement.
\item[Reference] INTDEP, section 3.4
\item[Learning objective] 3, 4, 7
\item[Main responsible] Linn\'ea Claesson
\end{description}

%\subsection*{Problem 7: Agile RE}
%\begin{description}
%\item[Proposition] Face-to-face communication between customer and the team is very common in an agile development project.
%\item[Reason] Benefits of face-to-face communication are: 1. Customer can steer the project in new directions. 2. Removes need for time-consuming documentation and approval processes.
%\item[Correct answer] A (Both the proposition and the reason are correct statements, AND the reason explains the proposition in a correct way.)
%\item[Motivation] According to the article, all 16 organizations rely heavily on face-to-face communication, the proposition is true. The reasons were reported by the participants, therefore the statement is true and explains the proposition in a correct way.
%\item[Reference] AGRE, p. 63 
%\item[Learning objective]
%\item[Main responsible] Linn\'ea Claesson
%\end{description}




\end{document}
% INNAN DU COMMITAR!
% Uppdatera datum
% Uppdatera version
%-----
% Document name



\documentclass[10pt,a4paper]{article}
\usepackage[utf8]{inputenc}
\usepackage[english]{babel}
\usepackage{amsmath}
\usepackage{amsfonts}
\usepackage{amssymb}
\usepackage{graphicx}
\usepackage{geometry}

\title{PostCardBuddy}
\author{Team C}

\begin{document}
\begin{titlepage}
\newgeometry{left=2cm,top=1cm,right=2cm}
\newcommand{\HRule}{\rule{\linewidth}{0.5mm}}


\begin{flushright}
November 16, 2015 v0.02\\[3cm]
\end{flushright}


\centering
\textsc{\LARGE Team C}\\[0.5cm]

\HRule \\[0.4cm]
{ \huge \bfseries PostCardBuddy}\\[0.3cm]
{\Large \bfseries Project Experiences}\\[0.4cm] % Title of your document
\HRule \\[1.5cm]

\vfill
\begin{flushleft}
%Authors, write on separate lines
\textit{Authors of this document:}\\
Emma Albertz\\
Caroline Brandberg\\
Linnéa Claesson\\
Billy Johansson\\
Johan Ju\\
Jacob Mejvik\\
Carl Rynegardh
\end{flushleft}

\end{titlepage}
\pagenumbering{gobble}



%\begin{center}
%\textit{\large Version History}
%
%    \begin{tabular}{ | l | l | l | p{5cm} |}
%    \hline
%    \textbf{Version} & \textbf{Date} & \textbf{Responsible} & \textbf{Description} \\ \hline
%    1.0 & 2015-10-14 & EA, LC & Baseline\\ \hline
%    \end{tabular}
%\end{center}



\setcounter{tocdepth}{2}
\tableofcontents
\newpage
\pagenumbering{arabic}

%---------------------------------------------------------------%
% A description of our requirements engineering work, including experiences and reflections in relation to learning objectives.
% The Project Experiences should not include course evaluation issues, but focus on your own work and learning outcome.
%---------------------------------------------------------------%

%-------------------------------------------------------------------%
%-------------- Background -----------------------------------------%
%-------------------------------------------------------------------%
\section{Background}


%-------------------------------------------------------------------%
%--------------- Methods and Techniques ----------------------------%
%-------------------------------------------------------------------%
% Description of the chosen methods/techniques for elicitation, specifcation, validation, and prioritization.
% Motivation for why you chose the used methods/techniques.

\section{Methods and Techniques}

% 3D) apply more than one elicitation technique in a relevant way.
\subsection{Elicitation}

% 4B) use at least four different specification techniques adequately tailored to the context.
\subsection{Specification}

% 3F) to assess the quality of requirements and find relevant problems of several different types.
% 3G) apply more than one validation technique.
% 4G) adapt the validation to the context and provide rationale for the chosen validation techniques.
\subsection{Validation}

% 3I) use more than one prioritization technique in a relevant way.
\subsection{Prioritization}

%--------------------------------------------------------------------%
%--------------- Reflections ----------------------------------------%
%--------------------------------------------------------------------%
% Reflection on the usage of these methods/techniques in terms of what was successful and what was challenging. Example questions for reflection: What have you learned in relation to the learning objectives in this course program? What would you have done differently if you would do this project again as a "real" project, based on what you know now? What have you learned in relation to the learning objectives?


% Osäker om detta är rätt placering: 5B) provide motivated estimations of target quality levels using well-defined scales.
% 4F) to find, prioritize and discuss requirements quality problems of different types, while reaching 	beyond form issues.
% 5D) reason about the relation between requirements quality problems and risks, both from a customer and developer viewpoint.
\section{Reflections}

% 3E) reflect on elicitation experiences.
% 4E) reason about the need for further elicitation in relation to specification quality.
% 5C) go beyond initial stakeholders and given frames, while challenging the domain boundaries and eliciting creative ideas and deep domain knowledge in real-world contexts.
\subsection{Elicitation}

% 3C) reflect on specification experiences and reason about choices of specification methods in relation to different contexts.
% 5A) combine specification techniques in an explicitly motivated trade off between qualities and costs, where a high degree of specification completeness is achieved for a carefully selected subset of requirements.
\subsection{Specification}

% 3H) reflect on validation experiences.
% 5E) utilize links among different types of specifications in validation efforts to find and address potentially harmful inconsistencies.
\subsection{Validation}

% 3J) reflect on prioritization experiences.
\subsection{Prioritization}

%--------------------------------------------------------------------%
%------------ Personal Statements -----------------------------------%
%--------------------------------------------------------------------%
% A personal statement by each team member that briefly explains each individual's contributions to the project results.
\section{Personal Statements}



\end{document}

